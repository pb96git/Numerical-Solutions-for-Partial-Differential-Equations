\documentclass[a4paper, 11pt]{article}
\usepackage{amsmath}
\usepackage{listings}
\usepackage{xcolor}
\usepackage{hyperref}
\usepackage{geometry}
\geometry{margin=1in}

\hypersetup{
	colorlinks=true,
	linkcolor=blue,
	filecolor=magenta,
	urlcolor=cyan,
}

\lstset{
	language=Python,                     % Set Python as the language
	basicstyle=\ttfamily\footnotesize,    % Set font size for the code
	keywordstyle=\color{blue},            % Python keywords in blue
	commentstyle=\color{gray},            % Comments in gray
	stringstyle=\color{red},              % Strings in red
	breaklines=true,                      % Enable line breaking
	breakatwhitespace=true,               % Allow breaks at whitespaces
	frame=single,                         % Frame the code
	columns=flexible,                     % Fix spaces between characters
	captionpos=b,                         % Caption at the bottom
	numbers=left,                         % Add line numbers
	numberstyle=\tiny\color{gray},        % Style of line numbers
	stepnumber=1,                         % Step between line numbers
	showspaces=false,                     % Don't show spaces
	showstringspaces=false                % Don't show string spaces
}

\title{Detailed Explanation of the Python Code: Wave Equation Solver}
\author{Petar Bosnic}
\date{October, 2024}

\begin{document}
	
	\maketitle
	
	\section{Introduction}
	
	This document explains the Python code for solving the 1D wave equation using various numerical methods, including scalar and vectorized implementations. The wave equation in its general form is:
	
	\begin{equation}
		u_{tt} = c^2 u_{xx} + f(x,t),
	\end{equation}
	where $u(x,t)$ is the displacement at position $x$ and time $t$, $c$ is the wave propagation speed, and $f(x,t)$ is an optional source term. The solver implements this equation over a spatial domain $(0, L)$ and a time domain $(0, T)$.
	
	
	\section{Understanding Time and Spatial Stepping in the Solver}
	
	To better understand how the wave equation solver updates values over time and space, we can visualize the computation as a grid. Each point on the grid represents a value of $u(x, t)$, where $x$ is a spatial point, and $t$ is a time step. The solver uses finite difference methods to compute values at each time step based on the previous ones.
	
	\subsection{2D Grid for Time and Spatial Stepping}
	
	The grid below represents the spatial domain on the $x$-axis and the time domain on the $t$-axis. The wave equation is solved iteratively, meaning that values at the next time step depend on the current and previous time steps.
	
	\begin{center}
		\begin{tabular}{|c|c|c|c|c|c|}
			\hline
			\textbf{Time Steps ($t$)} & \textbf{$x_0$} & \textbf{$x_1$} & \textbf{$x_2$} & \textbf{$\cdots$} & \textbf{$x_N$} \\
			\hline
			$t_0$ & $u_0^0$ & $u_1^0$ & $u_2^0$ & $\cdots$ & $u_N^0$ \\
			\hline
			$t_1$ & $u_0^1$ & $u_1^1$ & $u_2^1$ & $\cdots$ & $u_N^1$ \\
			\hline
			$t_2$ & $u_0^2$ & $u_1^2$ & $u_2^2$ & $\cdots$ & $u_N^2$ \\
			\hline
			$\vdots$ & $\vdots$ & $\vdots$ & $\vdots$ & $\ddots$ & $\vdots$ \\
			\hline
			$t_n$ & $u_0^n$ & $u_1^n$ & $u_2^n$ & $\cdots$ & $u_N^n$ \\
			\hline
		\end{tabular}
	\end{center}
	
	\textbf{Explanation}:
	\begin{itemize}
		\item Each row represents a time step $t_n$.
		\item Each column represents a spatial point $x_i$.
		\item The value $u_i^n$ represents the solution at spatial point $x_i$ and time step $t_n$.
		\item The solver updates the values $u_i^n$ using the finite difference scheme, based on previous time steps.
	\end{itemize}
	
	\subsection{Finite Difference Update Rule}
	
	At each time step, the values $u_i^n$ are updated using the values from previous time steps. For example, using the centered finite difference method, the update rule is:
	
	\begin{equation}
		u_i^{n+1} = 2u_i^n - u_i^{n-1} + C^2 (u_{i-1}^n - 2u_i^n + u_{i+1}^n)
	\end{equation}
	
	\textbf{Explanation of Terms}:
	\begin{itemize}
		\item $u_i^{n+1}$: The value at the next time step.
		\item $u_i^n$: The value at the current time step.
		\item $u_i^{n-1}$: The value from the previous time step.
		\item $C^2$: The square of the Courant number, which controls the stability of the numerical scheme.
		\item $u_{i-1}^n$, $u_{i+1}^n$: The neighboring spatial points used for computing the spatial derivative.
	\end{itemize}
	
	\subsection{Visualization of the Stepping Process}
	
	The update rule uses values from the current time step and the previous time step to compute the values at the next time step. Visually, the grid looks like this:
	
	\begin{center}
		\begin{tabular}{|c|c|c|}
			\hline
			& $u_i^{n-1}$ & \\
			\hline
			$u_{i-1}^n$ & $u_i^n$ & $u_{i+1}^n$ \\
			\hline
			& $u_i^{n+1}$ & \\
			\hline
		\end{tabular}
	\end{center}
	
	\textbf{Explanation}:
	\begin{itemize}
		\item The solver uses the values at $u_i^n$, $u_i^{n-1}$, and the neighboring spatial points $u_{i-1}^n$, $u_{i+1}^n$ to compute the new value $u_i^{n+1}$.
		\item This process is repeated for all spatial points and for each time step until the entire grid is filled with the solution.
	\end{itemize}
	
	\subsection{Conclusion}
	
	This section provides a visual representation of the time and spatial stepping used in the wave equation solver. The finite difference method iteratively updates the grid, and the Courant number ensures the stability of the solution. This grid visualization helps to understand how each point in the solution is computed based on previous values and neighboring points.
	
	\section{Code}
	\subsection{Function Definition and Parameters}
	
	The Python function \texttt{solve\_wave\_equation} is responsible for solving the wave equation numerically. Below is the function definition, where each parameter is described:
	
	\lstset{language=Python}
	\begin{lstlisting}
		def solve_wave_equation(I, V, f, c, L, dt, C, T, user_action=None, version='scalar', save_dir=None):
		"""
		Unified solver: Solve u_tt = c^2 * u_xx + f on (0, L) x (0, T] using scalar or vectorized approaches.
		"""
	\end{lstlisting}
	
	\textbf{Parameters}:
	\begin{itemize}
		\item \texttt{I}: The initial displacement function $I(x)$ that specifies the wave’s initial configuration.
		\item \texttt{V}: The initial velocity function $V(x)$ that defines the initial speed of the wave.
		\item \texttt{f}: The source term $f(x,t)$, which represents external forces acting on the system. This can be zero.
		\item \texttt{c}: The wave speed, which governs how fast the wave propagates through the domain.
		\item \texttt{L}: The length of the spatial domain $(0, L)$.
		\item \texttt{dt}: The time step size.
		\item \texttt{C}: The Courant number, defined as $C = \frac{c dt}{dx}$. This value controls the stability of the solver.
		\item \texttt{T}: The total time for the simulation.
		\item \texttt{user\_action}: A user-defined function that can be applied at each time step (optional).
		\item \texttt{version}: Specifies whether the solver uses a scalar or vectorized implementation.
		\item \texttt{save\_dir}: The directory to save results, such as images of the wave, during the simulation.
	\end{itemize}
	
	\subsection{Grid Setup and Courant Number}
	
	The grid setup initializes the computational domain in both space and time. The Courant number $C$ is squared for use later in the finite difference scheme.
	
	\lstset{language=Python}
	\begin{lstlisting}
		if save_dir is None:
		save_dir = os.getcwd()  # Save images in the current working directory
		
		Nt = int(round(T / dt))  # Number of time steps
		t = np.linspace(0, Nt * dt, Nt + 1)  # Time mesh points
		dx = dt * c / float(C)  # Spatial step size based on Courant number
		Nx = int(round(L / dx))  # Number of spatial points
		x = np.linspace(0, L, Nx + 1)  # Spatial mesh points
		C2 = C ** 2  # Courant number squared
		
		print("Courant number is:", C)
	\end{lstlisting}
	
	\textbf{Explanation}:
	\begin{itemize}
		\item The number of time steps \texttt{Nt} is computed based on the total time $T$ and the time step size $dt$.
		\item The spatial grid points \texttt{x} and the time grid points \texttt{t} are generated.
		\item The spatial step size \texttt{dx} is calculated using the Courant number $C$ to ensure stability.
		\item \texttt{C2} is the square of the Courant number $C$, which will be used in the time-stepping loop.
	\end{itemize}
	
	\subsection{Initialization of Solution Arrays}
	
	Here, the code initializes arrays to store the solution at the current time step (\texttt{u}), the previous time step (\texttt{u\_1}), and two time steps ago (\texttt{u\_2}).
	
	\lstset{language=Python}
	\begin{lstlisting}
		u = np.zeros(Nx + 1)
		u_1 = np.zeros(Nx + 1)
		u_2 = np.zeros(Nx + 1)
		
		t0 = time.time()
		
		for i in range(Nx + 1):
		u_1[i] = I(x[i])
	\end{lstlisting}
	
	\textbf{Explanation}:
	\begin{itemize}
		\item \texttt{u}, \texttt{u\_1}, and \texttt{u\_2} are initialized as arrays of zeros, representing the wave at different time steps.
		\item \texttt{u\_1} is filled with the initial displacement function $I(x)$, which represents the initial configuration of the wave.
		\item The time \texttt{t0} is recorded to calculate the CPU time later.
	\end{itemize}
	
	\subsection{First Time Step Calculation}
	
	The first time step is special because it requires both the initial displacement and velocity to initialize the solution.
	
	\lstset{language=Python}
	\begin{lstlisting}
		n = 0
		for i in range(1, Nx):
		u[i] = u_1[i] + dt * V(x[i]) + \
		0.5 * C2 * (u_1[i - 1] - 2 * u_1[i] + u_1[i + 1]) + \
		0.5 * dt ** 2 * f(x[i], t[n])
		u[0] = 0
		u[Nx] = 0
	\end{lstlisting}
	
	\textbf{Explanation}:
	\begin{itemize}
		\item The code computes the new solution at the first time step using a combination of the initial displacement \texttt{u\_1}, velocity \texttt{V}, and the source term \texttt{f(x,t)}.
		\item A second-order accurate finite difference scheme is used.
		\item Boundary conditions are applied by setting \texttt{u[0]} and \texttt{u[Nx]} to zero (Dirichlet boundary conditions).
	\end{itemize}
	
	\subsection{Time-Stepping Loop}
	
	The core of the numerical solver is the time-stepping loop, which advances the solution in time using the scalar or vectorized approach.
	
	\lstset{language=Python}
	\begin{lstlisting}
		for n in range(1, Nt):
		if version == 'scalar':
		for i in range(1, Nx):
		u[i] = -u_2[i] + 2 * u_1[i] + \
		C2 * (u_1[i - 1] - 2 * u_1[i] + u_1[i + 1]) + \
		dt ** 2 * f(x[i], t[n])
		
		elif version == 'vectorized':
		f_a = f(x, t[n])
		u[1:-1] = -u_2[1:-1] + 2 * u_1[1:-1] + \
		C2 * (u_1[0:-2] - 2 * u_1[1:-1] + u_1[2:]) + \
		dt ** 2 * f_a[1:-1]
	\end{lstlisting}
	
	\textbf{Explanation}:
	\begin{itemize}
		\item In the scalar version, each point in the domain is updated sequentially using the finite difference method.
		\item In the vectorized version, the entire domain is updated simultaneously using NumPy’s slicing, which makes the code more efficient.
		\item The source term \texttt{f} is applied at each time step.
	\end{itemize}
	
	\subsection{Boundary Conditions and User Action}
	
	Boundary conditions are enforced after each time step, and the optional \texttt{user\_action} is called to perform actions such as saving images or logging results.
	
	\lstset{language=Python}
	\begin{lstlisting}
		u[0] = 0
		u[Nx] = 0
		
		if user_action is not None:
		user_action(u, x, t, n + 1)
		
		u_2[:] = u_1
		u_1[:] = u
	\end{lstlisting}
	
	\textbf{Explanation}:
	\begin{itemize}
		\item Boundary conditions are applied at every time step to ensure the wave behavior is consistent with physical constraints.
		\item \texttt{user\_action} allows flexibility for additional tasks such as saving the current state of the solution or visualizing the wave propagation.
		\item The arrays \texttt{u\_2} and \texttt{u\_1} are updated to prepare for the next time step.
	\end{itemize}
	
	\subsection{Finalizing the Simulation and CPU Time}
	
	The simulation ends by calculating the CPU time, which helps in evaluating the performance of the solver.
	
	\lstset{language=Python}
	\begin{lstlisting}
		cpu_time = time.time() - t0
		print(f"CPU time (s) of {version} solver: {cpu_time:.5f}")
		return u, x, t, cpu_time
	\end{lstlisting}
	
	\textbf{Explanation}:
	\begin{itemize}
		\item The CPU time for the entire simulation is calculated and printed.
		\item The final wave solution \texttt{u}, the spatial grid \texttt{x}, the time grid \texttt{t}, and the CPU time are returned.
	\end{itemize}
	
	\subsection{Conclusion}
	
	This Python function provides a unified approach to solving the 1D wave equation using either scalar or vectorized methods. The use of the Courant number ensures stability, and the code is flexible, allowing for different types of user-defined actions during the simulation.

\section{Visualization of the Simulation}

In addition to solving the wave equation, the code provides functionality to visualize the wave propagation by saving images at each time step and generating a GIF of the entire simulation. The following functions handle the visualization process.

\subsection{Saving Wave Images}

The \texttt{save\_wave\_image} function saves a snapshot of the wave at each time step. The Courant number $C$ is included in the plot title for reference.

\lstset{language=Python}
\begin{lstlisting}
	def save_wave_image(u, x, t, n, C, save_dir='wave_images'):
	"""Save the wave at each time step as an image, with Courant number in the title."""
	if not os.path.exists(save_dir):
	os.makedirs(save_dir)
	
	plt.figure(figsize=(8, 4))
	plt.plot(x, u, label=f"t = {t[n]:.5f}")
	plt.ylim(-0.01, 0.01)
	plt.xlim(0, max(x))
	plt.xlabel('x')
	plt.ylabel('u(x,t)')
	
	# Include Courant number in the plot title
	plt.title(f"Wave propagation at time t = {t[n]:.5f}, Courant number = {C}")
	
	plt.legend()
	plt.grid(True)
	filename = os.path.join(save_dir, f'wave_step_{n:04d}.png')
	
	# Debugging message
	print(f"Saving image: {filename}")
	
	plt.savefig(filename)
	plt.close()
\end{lstlisting}

\textbf{Explanation}:
\begin{itemize}
	\item The function creates a directory (\texttt{save\_dir}) for storing the images, if it does not already exist.
	\item A plot of the wave at the current time step is generated, and the Courant number is displayed in the title.
	\item Each image is saved with a filename that includes the time step number, ensuring proper ordering.
\end{itemize}

\subsection{Generating a GIF}

The \texttt{generate\_gif\_from\_images} function creates an animated GIF from the saved images, allowing you to visualize the entire wave propagation over time.

\lstset{language=Python}
\begin{lstlisting}
	def generate_gif_from_images(image_folder='wave_images', gif_name='wave_animation.gif', duration=0.1):
	"""Generate a GIF from the images stored in a folder."""
	images = []
	image_files = sorted([img for img in os.listdir(image_folder) if img.endswith(".png")])
	
	if not image_files:
	print("Warning: No images found in the specified folder to create a GIF.")
	return
	
	for filename in image_files:
	image_path = os.path.join(image_folder, filename)
	images.append(imageio.imread(image_path))
	
	gif_path = os.path.join(image_folder, gif_name)
	imageio.mimsave(gif_path, images, duration=duration)
	print(f"GIF saved as {gif_path}")
\end{lstlisting}

\textbf{Explanation}:
\begin{itemize}
	\item The function gathers all the PNG files in the specified folder and sorts them to ensure proper sequence.
	\item The images are combined into a GIF using \texttt{imageio.mimsave}, with a specified frame duration for smooth animation.
	\item The resulting GIF is saved in the folder, and its path is printed for reference.
\end{itemize}

\subsection{Conclusion}

This section of the code provides functionality to visually track the wave's propagation during the simulation. The ability to generate a GIF allows for easy visualization of the wave dynamics over time.

\section{Main Program}

This section provides the main program that sets up the wave equation simulation, calls the solver, and generates visualizations. The program is divided into logical steps for clarity.

\subsection{Importing Required Functions}

First, the necessary functions from the \texttt{wave\_eq\_solver} module are imported.

\lstset{language=Python}
\begin{lstlisting}
	import numpy as np
	from wave_eq_solver import solve_wave_equation, save_wave_image, generate_gif_from_images
\end{lstlisting}

\textbf{Explanation}:
\begin{itemize}
	\item \texttt{solve\_wave\_equation}: Solves the wave equation.
	\item \texttt{save\_wave\_image}: Saves images of the wave propagation at each time step.
	\item \texttt{generate\_gif\_from\_images}: Generates a GIF from saved wave images.
\end{itemize}

\subsection{Defining Parameters}

The parameters for the wave equation are defined based on the problem of a vibrating guitar string.

\lstset{language=Python}
\begin{lstlisting}
	if __name__ == "__main__":
	
		# Parameters for the wave equation
		L = 0.75                # Length of the string (in meters)
		x0 = 0.8 * L            # Initial displacement position
		a = 0.005               # Maximum amplitude of the initial displacement (in meters)
		freq = 440              # Frequency of the wave (in Hertz)
		wavelength = 2 * L       # Wavelength of the wave
		c = freq * wavelength    # Wave speed
		omega = 2 * np.pi * freq # Angular frequency
		num_periods = 1          # Number of periods to simulate
		T = 2 * np.pi / omega * num_periods   # Total simulation time (in seconds)
		C = 0.2                 # Courant number
		dt = L / 50.0 / c       # Time step size (in seconds)
\end{lstlisting}

\textbf{Explanation}:
\begin{itemize}
	\item \texttt{L}: The length of the string is set to 0.75 meters.
	\item \texttt{freq}: The frequency of the string is 440 Hz (standard A note).
	\item \texttt{c}: The wave speed is computed using the wave frequency and wavelength.
	\item \texttt{C}: The Courant number controls the stability of the simulation. It is set to 0.2 for this example.
	\item \texttt{T} and \texttt{dt}: The total simulation time is set to cover one period of oscillation, and \texttt{dt} is computed for proper resolution.
\end{itemize}

\subsection{Defining Initial Conditions and Source Term}

The initial displacement, velocity, and source term functions are defined. These are specific to the vibrating string problem.

\lstset{language=Python}
\begin{lstlisting}
	# Problem-specific functions for the guitar string
	def initial_displacement(x):
	"""Initial condition for displacement."""
	return a * x / x0 if x < x0 else a / (L - x0) * (L - x)
	
	def initial_velocity(x):
	"""Initial condition for velocity (e.g., zero velocity)."""
	return np.zeros_like(x)
	
	def source_term(x, t):
	"""Optional source term f(x, t), here assumed zero."""
	return np.zeros_like(x)
\end{lstlisting}

\textbf{Explanation}:
\begin{itemize}
	\item \texttt{initial\_displacement}: Models the initial pluck of the string, where the displacement is a triangular shape.
	\item \texttt{initial\_velocity}: Sets the initial velocity to zero (the string starts at rest).
	\item \texttt{source\_term}: No external forces are applied in this case, so the source term is set to zero.
\end{itemize}

\subsection{Choosing the Solver}

The user is prompted to choose between the scalar or vectorized solvers. This allows the user to control the performance and behavior of the simulation.

\lstset{language=Python}
\begin{lstlisting}
	# Ask user to choose between scalar or vectorized solver
	solver_choice = input("Choose solver (scalar/vectorized/vectorized2): ").strip()
\end{lstlisting}

\textbf{Explanation}:
\begin{itemize}
	\item The user can choose between three solver versions: \texttt{scalar}, \texttt{vectorized}, and \texttt{vectorized2}.
	\item The \texttt{scalar} version uses a simple loop, while the \texttt{vectorized} versions use efficient array operations for faster computations.
\end{itemize}

\subsection{Solving the Wave Equation}

The \texttt{solve\_wave\_equation} function is called to solve the wave equation based on the user’s choice of solver.

\lstset{language=Python}
\begin{lstlisting}
	# Use the unified solver based on user choice
	if solver_choice in ["scalar", "vectorized", "vectorized2"]:
	solve_wave_equation(
	initial_displacement, 
	initial_velocity, 
	source_term, 
	c, 
	L, 
	dt, 
	C, 
	T, 
	user_action=lambda u, x, t, n, save_dir: save_wave_image(u, x, t, n, C, save_dir),  # Pass C via lambda
	version=solver_choice,
	save_dir='wave_images'
	)
	else:
	print("Invalid choice. Please choose 'scalar', 'vectorized', or 'vectorized2'.")
\end{lstlisting}

\textbf{Explanation}:
\begin{itemize}
	\item \texttt{solve\_wave\_equation}: This function solves the wave equation using the initial conditions, velocity, and source term defined earlier.
	\item \texttt{user\_action}: A lambda function is passed to save images of the wave at each time step. The Courant number $C$ is passed to the \texttt{save\_wave\_image} function.
	\item \texttt{version}: The user’s choice of solver (\texttt{scalar}, \texttt{vectorized}, or \texttt{vectorized2}) determines which method is used to solve the equation.
\end{itemize}

\subsection{Generating a GIF}

After solving the wave equation and saving the images, a GIF is generated from the saved images to visualize the wave propagation over time.

\lstset{language=Python}
\begin{lstlisting}
	# Generate a GIF from the saved images
	generate_gif_from_images(image_folder='wave_images', gif_name='wave_animation.gif', duration=0.1)
\end{lstlisting}

\textbf{Explanation}:
\begin{itemize}
	\item The function \texttt{generate\_gif\_from\_images} creates an animated GIF of the wave propagation. The frame duration is set to 0.1 seconds.
	\item The images are taken from the \texttt{wave\_images} folder, where they were saved during the simulation.
\end{itemize}

\subsection{Conclusion}

This section provides the main program for simulating the wave equation. It allows the user to select between different solvers, visualize the wave propagation as images, and generate an animated GIF to capture the entire process. The solver is flexible, allowing custom initial conditions, and the visualization helps track the evolution of the wave over time.



	
\end{document}
